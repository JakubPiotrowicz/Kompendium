\documentclass[zad,zawodnik,utf8]{sinol}
\usepackage{amsfonts}
\usepackage{amsmath}
\usepackage{enumitem}
\usepackage{braket}
\usepackage{xcolor}
	\title{Kontest 1 - Starsi}
	\id{mor}
	\signature{xxx0000}
	\author{XXX YYYY}

	\pagestyle{fancy}
	\iomode{stdin}
 	\konkurs{Rudki 27.09.2022}
	
	\day{26.09.2022}
	\date{Kontest 1}

\begin{document}
    \begin{center}
        \textbf{\huge Kontest 1 - 27.09.2022} \\ 
        \vspace{0.5cm}
        \Large{Starsi}
    \end{center}
    \Large{
        \textbf{Zadanie 1.} Niech $AA'$ będzie środkową w trójkącie $ABC$. Niech $D$ będzie punktem na $AA'$ oraz $E$ przecięciem $BD$ i $AC$. Okrąg opisany na trójkącie $BCE$ przecina $AB$ ponownie w punkcie $F$. \\ Udowodnij, że jeśli punkty $C$, $D$ i $F$ są współliniowe to trójkąt $ABC$ jest równoramienny. \\
        
        \textbf{Zadanie 2.} Dany jest graf dwudzielny o częściach $A$ i $B$, w którym $|A|=2n$, $|B|=2n+1$ oraz wszystkie wierzchołki w części $A$ mają ten sam stopień. \\
        Wykaż, że pewne dwa wierzchołki w części $B$ mają ten sam stopień. \\
        
        \textbf{Zadanie 3.}  Niech $AD,BE,CF$ będą wysokościami w trójkącie ostrokątnym $ABC$. Prosta równoległa do $EF$ przechodząca przez $D$ przecina prostą $AB$ w punkcie $R$ i $AC$ w $Q$. Niech $P$ będzie przecięciem prostych $EF$ i $CB$. \\ 
        Udowodnij, że okrąg opisany na $PQR$ przechodzi przez środek odcinka $BC$. \\
        
        \textbf{Zadanie 4.} Wyznacz liczbę par dodatnich liczb całkowitych $m,n$ spełniających równanie: $$m(m+1)=(n-17)(n+17)$$.
    }
\end{document}
