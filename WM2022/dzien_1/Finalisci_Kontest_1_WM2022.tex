\documentclass[zad,zawodnik,utf8]{sinol}
\usepackage{amsfonts}
\usepackage{amsmath}
\usepackage{enumitem}
\usepackage{braket}
\usepackage{xcolor}
	\title{Kontest 1 - Finaliści}
	\id{mor}
	\signature{xxx0000}
	\author{XXX YYYY}

	\pagestyle{fancy}
	\iomode{stdin}
 	\konkurs{Rudki 27.09.2022}
	
	\day{26.09.2022}
	\date{Kontest 1}

\begin{document}
    \begin{center}
        \textbf{\huge Kontest 1 - 27.09.2022} \\ 
        \vspace{0.5cm}
        \Large{Finaliści}
    \end{center}
    \Large{
        \textbf{Zadanie 1.} Udowodnij, że jeżeli wielomian $f(x) = x^6 + ax^3 + bx^2 + cx + d$ ma $6$ pierwiastków rzeczywistych (z krotnościami), to $a = b = c = d = 0$. \\
        
        \textbf{Zadanie 2.} Rozstrzygnij czy istnieje na płaszczyźnie niepusty, skończony zbiór okręgów o rozłącznych wnętrzach takich, że każdy jest styczny do dokładnie pięciu z pozostałych. \\
        
        \textbf{Zadanie 3.}  Niech $ABC$ będzie trójkątem ostrokątnym oraz $M$ środkiem boku $BC$ oraz $H$ ortocentrum. Odcinki $BE$ oraz $CF$ są wysokościami w trójkącie $ABC$. Przypuśćmy, że na prostej $EF$ jest taki punkt $X$, że $\angle XMH =\angle HAM$ oraz $A,X$ leżą po przeciwnych stronach $MH$. \\ 
        Udowodnij, że $AH$ przecina odcinek $MX$ w połowie. \\
        
        \textbf{Zadanie 4.} Niech $a_2,a_3,\ldots,a_n$ będą liczbami dodatnimi spełniającymi \\ 
        $a_2 \cdot a_3 \cdot \ldots \cdot a_n=1$. Udowodnij, że: $$(a_2+1)^2\cdot (a_3+1)^3\cdot \ldots \cdot (a_n+1)^n \geq n^n$$.
    }
\end{document}
