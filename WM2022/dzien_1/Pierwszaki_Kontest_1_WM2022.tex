\documentclass[zad,zawodnik,utf8]{sinol}
\usepackage{amsfonts}
\usepackage{amsmath}
\usepackage{enumitem}
\usepackage{braket}
\usepackage{xcolor}
	\title{Kontest 1 - Pierwszaki}
	\id{mor}
	\signature{xxx0000}
	\author{XXX YYYY}

	\pagestyle{fancy}
	\iomode{stdin}
 	\konkurs{Rudki 27.09.2022}
	
	\day{26.09.2022}
	\date{Kontest 1}

\begin{document}
    \begin{center}
        \textbf{\huge Kontest 1 - 27.09.2022} \\ 
        \vspace{0.5cm}
        \Large{Pierwszaki}
    \end{center}
    \Large{
        \textbf{Zadanie 1.} Udowodnij, że w dwunastokącie foremnym $A_1 A_2 ... A_{12}$ przekątne $A_1 A_5$, $A_3 A_8$ i $A_4 A_{11}$ przecinają się w jednym punkcie. \\
        
        \textbf{Zadanie 2.} Niech $a,b,c \in \mathbb{R}_+$ oraz $a+b+c \leq 4$ i $ab+bc+ca\geq 4$. \\
        Udowodnij, że conajmniej dwie z poniższych nierówności są prawdziwe:
        $$|a-b|\leq 2|,\hspace{0.1cm} |b-c|\leq 2,\hspace{0.1cm} |c-a|\leq 2$$
        
        \textbf{Zadanie 3.}  Niech $a$ i $b$ będą różnymi liczbami całkowitymi dodatnimi takimi, że $ab(a+b)$ jest podzielne przez $a^2+ab+b^2$. Udowodnij, że $|a-b| > \sqrt[3]{ab}$. \\
        
        \textbf{Zadanie 4.} Dany jest duży stosik kart. Na każdej karcie napisana jest jedna liczba ze zbioru $\{1,2,...,n\}$. Wiemy, że suma wszystkich liczb na kartach jest równa $k \cdot n!$ dla pewnego $k$. \\ Udowodnij, że możemy podzielić nasze karty na $k$ stosów tak, że suma każdego z nich jest równa $n!$.
    }
\end{document}
