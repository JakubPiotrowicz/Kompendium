\documentclass[12pt, a4paper]{article}
\usepackage[margin=2.5cm]{geometry}
\usepackage{fancyhdr}
\usepackage[MeX]{polski}
\usepackage[utf8]{inputenc} 
\usepackage[T1]{fontenc}
\usepackage{amsmath}
\usepackage{amsfonts}
\usepackage{amssymb}
\usepackage{amsthm}
\newcommand{\Z}{\mathbb{Z}}
\newcommand{\R}{\mathbb{R}}
\newcommand{\N}{\mathbb{N}}
\newcommand{\Q}{\mathbb{Q}}
\newcommand{\NWD}{\text{NWD}}
\newcommand{\NWW}{\text{NWW}}
\newcommand{\question}[1]{\normalitem \begin{samepage}#1 \end{samepage}}
\newcommand{\questionwithasterix}[1]{ \asterixitem \begin{samepage}#1 \vspace{6cm}\end{samepage}}
\cfoot{\thepage}

\begin{document}

\begin{center}
\noindent \textbf{Prosimy wypełnić poniższe pola DRUKOWANYMI literami:}
\vspace{0.5cm}
\par Imię i nazwisko
\par $\begin{array}{|c|c|c|c|c|c|c|c|c|c|c|c|c|c|c|c|c|c|c|c|c|c|c|c|c|c|c|c|c|c|c|c|c|c|c|c|}
\hline
\makebox[0.1cm]{}&\makebox[0.1cm]{}&\makebox[0.1cm]{}&\makebox[0.1cm]{}&\makebox[0.1cm]{}&\makebox[0.1cm]{}&\makebox[0.1cm]{}&\makebox[0.1cm]{}&\makebox[0.1cm]{}&\makebox[0.1cm]{}&\makebox[0.1cm]{}&\makebox[0.1cm]{}&\makebox[0.1cm]{}&\makebox[0.1cm]{}&\makebox[0.1cm]{}&\makebox[0.1cm]{}&\makebox[0.1cm]{}&\makebox[0.1cm]{}&\makebox[0.1cm]{}&\makebox[0.1cm]{}&\makebox[0.1cm]{}&\makebox[0.1cm]{}&\makebox[0.1cm]{}&\makebox[0.1cm]{}&\makebox[0.1cm]{}&\makebox[0.1cm]{}&\makebox[0.1cm]{}&\makebox[0.1cm]{}&\makebox[0.1cm]{}&\makebox[0.1cm]{}&\makebox[0.1cm]{}&\makebox[0.1cm]{}&\makebox[0.1cm]{}&\makebox[0.1cm]{}&\makebox[0.1cm]{}&\makebox[0.1cm]{}\\
\hline
\end{array}$
\vspace{0.2cm}
\par E-mail
\par $\begin{array}{|c|c|c|c|c|c|c|c|c|c|c|c|c|c|c|c|c|c|c|c|c|c|c|c|c|c|c|c|c|c|c|c|c|c|c|c|}
\hline
\makebox[0.1cm]{}&\makebox[0.1cm]{}&\makebox[0.1cm]{}&\makebox[0.1cm]{}&\makebox[0.1cm]{}&\makebox[0.1cm]{}&\makebox[0.1cm]{}&\makebox[0.1cm]{}&\makebox[0.1cm]{}&\makebox[0.1cm]{}&\makebox[0.1cm]{}&\makebox[0.1cm]{}&\makebox[0.1cm]{}&\makebox[0.1cm]{}&\makebox[0.1cm]{}&\makebox[0.1cm]{}&\makebox[0.1cm]{}&\makebox[0.1cm]{}&\makebox[0.1cm]{}&\makebox[0.1cm]{}&\makebox[0.1cm]{}&\makebox[0.1cm]{}&\makebox[0.1cm]{}&\makebox[0.1cm]{}&\makebox[0.1cm]{}&\makebox[0.1cm]{}&\makebox[0.1cm]{}&\makebox[0.1cm]{}&\makebox[0.1cm]{}&\makebox[0.1cm]{}&\makebox[0.1cm]{}&\makebox[0.1cm]{}&\makebox[0.1cm]{}&\makebox[0.1cm]{}&\makebox[0.1cm]{}&\makebox[0.1cm]{}\\
\hline
\end{array}$
\vspace{0.2cm}
\par \hspace{1.7cm} Nr telefonu \hspace{2.5cm} Klasa
\par $\begin{array}{|c|c|c|c|c|c|c|c|c|c|c|c|}
\hline
\makebox[0.1cm]{+}&\makebox[0.1cm]{4}&\makebox[0.1cm]{8}&\makebox[0.1cm]{}&\makebox[0.1cm]{}&\makebox[0.1cm]{}&\makebox[0.1cm]{}&\makebox[0.1cm]{}&\makebox[0.1cm]{}&\makebox[0.1cm]{}&\makebox[0.1cm]{}&\makebox[0.1cm]{}\\
\hline
\end{array}$
\hspace{1cm}
$\begin{array}{|c|c|}
\hline
\makebox[0.1cm]{}&\makebox[0.1cm]{}\\
\hline
\end{array}$
\end{center}\vspace{0.5cm}
	\begin{center}{\large{\hspace{1cm}\textbf{Test kwalifikacyjny na Warsztaty Matematyczne 2022}}}
				\newline \newline Klasy trzecie i czwarte\end{center}
\par
\vspace{0.3cm}
Test składa się z uporządkowanych w kolejności \underline{\textbf{losowej}} 30 zestawów po 3 pytania. Na pytania odpowiada się ,,tak'' lub ,,nie'' poprzez wpisanie odpowiednio ,,\textbf{T}'' bądź ,,\textbf{N}''
w pole obok pytania. W danym trzypytaniowym zestawie możliwa jest dowolna kombinacja
odpowiedzi ,,tak'' i ,,nie''. W zestawach zaznaczonych gwiazdką (gwiazdka wygląda tak: *)
prócz udzielenia odpowiedzi należy je uzasadnić. Test trwa 180 minut.
\vspace{0.5cm}
\par
\textbf{Zasady punktacji}
\begin{itemize}
   \item Za pojedynczą poprawną odpowiedź: \textbf{1} punkt.
   \item Za pojedynczą niepoprawną odpowiedź: \textbf{-1} punkt.
   \item Za brak odpowiedzi: \textbf{0} punktów.
   \item Za zadanie zrobione w całości dobrze dodatkowe \textbf{2} punkty.
   \item Za poprawne uzasadnienie pojedynczej odpowiedzi: \textbf{1} punkt.
   \item Za niepoprawne uzasadnienie pojedynczej odpowiedzi bądź brak takowego: \textbf{0} pkt.
\end{itemize}
	\begin{center}Powodzenia!\end{center}
Uwaga! Przez zbiór liczb naturalnych w zadaniach rozumiemy zbiór liczb całkowitych większych lub równych $0$.
\vspace{0.2cm}

\renewcommand{\labelitemi}{\LARGE{$\square$}}
\newcommand{\normalitem}{\stepcounter{enumi}\item[\textbf{\theenumi. }]}
\newcommand{\asterixitem}{\stepcounter{enumi}\item[\textbf{\theenumi*. }]}

\begin{enumerate}
    \question {
		Czy istnieje $100$ kolejnych liczb naturalnych wśród których:
	
		\begin{itemize}
			\item Dokładnie $12$ jest liczbami Fibbonacciego?
			\item Dokładnie $7$ jest liczbami pierwszymi?
			\item Dokładnie $7$ jest potęgami dwójki o całkowitym wykładniku?
		\end{itemize}
		
	}
	
	\questionwithasterix {
		Wielomian $W(x)$ ma współczynniki rzeczywiste oraz $W(8) = 8$, $W(-4) = -8$. Wiadomo, że istnieje wielomian $P(x)$ o współczynnikach rzeczywistych taki, że $W(x) = x \cdot P(x^2)$.
	
		\begin{itemize}
			\item Na pewno $W(0) = 0$.
			\item Na pewno $W(8) + W(-8) = 0$.
			\item Na pewno $W(2) = 0$.
		\end{itemize}
	}
	
	\questionwithasterix{
		Niech $A=(0,0)$, $B=(1,0)$, $C=(2,0)$, $D=(3,0)$, $E=(0,1)$. Czy
	
		\begin{itemize}
			\item $\measuredangle BED = \measuredangle ECA$?
			\item $\measuredangle BEC = \measuredangle EDA$?
			\item $\measuredangle AEB = \measuredangle EBA$?
		\end{itemize}
	}
	
	\question{
		Rozpatrzmy ciąg $a$, $a+b$, $a+2b$, $a+3b$, $\ldots$
	
		\begin{itemize}
			\item Dla $a=720$, $b=7$ są w nim co najmniej $4$ liczby pierwsze.
			\item Dla $a=39$, $b=57$ jest w nim liczba pierwsza.
			\item Dla $a=8953$, $b=22$ jest w nim co najmniej $5$ liczb pierwszych.
		\end{itemize}
	}
	
	\question{
		Niech $d(n)$ oznacza sumę wszystkich dodatnich dzielników liczby naturalnej $n$ i $D(n) = d(1) + d(2) + \ldots + d(n)$.
	
		\begin{itemize}
			\item $d(840) \leq 8400$.
			\item $d(1 \cdot 3 \cdot \ldots \cdot (2n+1)) = d(1) \cdot d(3) \cdot \ldots \cdot d(2n+1)$.
			\item $D(32) \geq 1024$.
		\end{itemize}
	}
	
	\question {
		W klasie w matexie są $23$ osoby, w tym $2$ dziewczyny, $10$ uczestników OM i $20$ graczy brydża (grupy są niezależne od siebie). Najbardziej prawdopodobne jest, że...
		
		\begin{itemize}
			\item dwójka najlepszych graczy w brydża w klasie jest dziewczynami.
			\item obie dziewczyny startują w OM i wszyscy uczestnicy OM w tej klasie grają w brydża.
			\item pewne dwie osoby obchodzą urodziny tego samego dnia (przyjmij że rok ma 365 dni, lata przestępne nie istnieją i każdy dzień w roku ma równe prawdopodobieństwo bycia dniem urodzin).
		\end{itemize}
	}
	
	\questionwithasterix {
		Niech $e_1 = 2$ i dla $n>0$ zachodzi $e_{n+1} = e_1 \cdot \ldots \cdot e_n + 1$.
    
    	\begin{itemize}
			\item Dla $a \ne b$ : $NWD(e_a, e_b) = 1$.
			\item Ostatnią cyfrą $e_{2022}$ jest $1$.
			\item $e_{100}$ ma więcej niż $2^{100}$ cyfr.
		\end{itemize}
	}
	
	\question {
		Wielomian $P(x) = 3x^5+x^4+7x^3-2x-1$ posiada pierwiastek:
	
		\begin{itemize}
			\item całkowity.
			\item wymierny.
			\item rzeczywisty.
		\end{itemize}
	}
	
	\question {
		Liczba $1000000000601$ jest:
	
		\begin{itemize}
			\item kwadratem liczby całkowitej.
			\item sześcianem liczby całkowitej.
			\item liczbą pierwszą.
		\end{itemize}
	}
	
	\questionwithasterix {
		Równanie $x^2+11y^2=kz^2$ ma nieskończenie wiele rozwiązań w liczbach całkowitych dla:
	
		\begin{itemize}
			\item $k=5$.
			\item $k=6$.
			\item nieskończenie wielu całkowitych $k$.
		\end{itemize}
	}
	
	\question{
		Dany jest sześciokąt wypukły $ABCDEF$, w którym $\measuredangle  EFA = \measuredangle FAB = \measuredangle ABC = \measuredangle BCD = 120^{\circ}$ i $\measuredangle CFE = \measuredangle FCD$.
	
		\begin{itemize}
			\item $ABCDEF$ jest foremny.
			\item $AF = BC$.
			\item Na $ABCF$ można opisać okrąg.
		\end{itemize}
	}
	
	\question{
		 Martyna i Oliwia grają w grę. Ruch polega na zamianie liczby całkowitej $n$ na dowolną liczbę całkowitą z przedziału $[\frac{n}{4},\frac{n}{2}]$. Przegrywa ta, która nie może wykonać ruchu. Martyna zaczyna.
	
		\begin{itemize}
			\item Dla $n=100$ Martyna ma strategię wygrywającą.
			\item Wśród liczb $[1,1000]$ jest dokładnie $620$ dających strategię wygrywającą dla Martyny.
			\item Dla $n=10^6$ Martyna nie ma strategii wygrywającej.
		\end{itemize}
	}
	
	\question{
		Mamy dany trójkąt równoboczny $ABC$ o polu $7$. Punkty $M$ i $N$ leżą odpowiednio na bokach $AB$ i $AC$, że $AN=BM$. Punkt $O$ jest przecięciem $BN$ i $CM$, a pole $BOC$ jest równe $2$.
	
		\begin{itemize}
			\item Kąt $BOC$ jest równy $120^{\circ}$.
			\item Stosunek $\frac {MB}{AB}$ może być równy $\frac{1}{3}$.
			\item Kąt $AOB$ jest równy $150^{\circ}$.
		\end{itemize}
	}
	
	\question {
		Rozważmy ciąg rekurencyjny o wzorze $a_{n+3}=a_{n+2} \cdot a_{n+1}+a_{n}$ oraz $a_1=a_2=a_3=1$.
	
		\begin{itemize}
			\item Istnieje taki indeks $m$, że $a_m=1000$.
			\item Istnieje nieskończenie wiele liczb w tym ciągu podzielnych przez $3$.
			\item Dla każdej liczby całkowitej $n$ istnieje w tym ciągu liczba będąca wielokrotnością $n$.
		\end{itemize}
	}
	
	\questionwithasterix {
		Mamy stosy kamieni. Dwóch graczy na przemian zabiera dowolną liczbę kamieni z najliczniejszego stosu. Wygrywa ten gracz, po którego ruchu stół zostanie pusty. Czy w podanych układach stosów gracz pierwszy może zawsze wygrać, niezależnie od ruchów przeciwnika?
	
		\begin{itemize}
			\item $7,7,3,2,2,1$
			\item $6,6,6,6,6$
			\item $3,3,3,2,2,2,1,1,1$ 
		\end{itemize}
	}
	
	\question {
		Dane są trzy okręgi o środkach $O_1$, $O_2$, $O_3$ i promieniach odpowiednio $3$, $4$ i $21$, takie że każde dwa z nich są zewnętrznie styczne. W trójkąt $O_1 O_2 O_3$ wpisano okrag $\omega$ o środku $O$.
	
		\begin{itemize}
			\item Promień $\omega$ wynosi $3$.
			\item Długość odcinka $OO_2$ jest mniejsza niż $\frac{9}{2}$.
			\item Pole trójkąta $O_1O_2O_3$ jest większe od $81$.
		\end{itemize}
	}
	
	\question {
		Pokolorujmy wszystkie liczby całkowite dodatnie na $2$ kolory. Czy:
	
		\begin{itemize}
			\item zawsze istnieje jednokolorowy niestały ciąg arytmetyczny długości $3$?
			\item istnieje takie kolorowanie, że suma dwóch dowolnych różnych jednokolorowych liczb nie jest potęgą dwójki?
			\item zawsze istnieją takie parami różne jednokolorowe liczby $x, y, z$ że $x+y=z$?
		\end{itemize}
	}
	
	\question {
		Dany jest turniej - każdy zawodnik rozgrywa dokładnie jeden mecz z każdym innym i nie ma remisów. Cyklem $k$-elementowym nazwiemy taki ciąg parami różnych zawodników, że pierwszy wygrywa z drugim, drugi z trzecim itd., aż na końcu $k$-ty wygrywa z pierwszym. Zawsze prawdą jest, że:
	
			\begin{itemize}
			\item jeśli każdy zawodnik zwyciężył z $k$ innymi to istnieje cykl co najmniej $(k+2)$-elementowy.
			\item jeśli każdy zawodnik z kimś wygrał i nikt nie wygrał z każdym to zawsze istnieje cykl.
			\item jeśli w turnieju złożonym z $n \geq 3$ zawodników istnieje cykl $n$-elementowy to istnieje także $3$-elementowy.
		\end{itemize}
	}
	
	\question {
		Które z następujących zdań są równoważne zdaniu: "Jeżeli $p$ jest prawdziwe, wtedy $q$ jest fałszywe"?
	
		\begin{itemize}
			\item Jeżeli $q$ jest fałszywe, to $p$ jest prawdziwe. 
			\item Jeżeli $q$ jest prawdziwe, to $p$ jest fałszywe.
			\item Albo oba $p$ i $q$ są fałszywe, albo dokładnie jedno z nich jest fałszywe. 
		\end{itemize}
	}
	
	\question {
		Konstruujemy ciąg $a_n$, w którym $0 \leq a_1,a_2 < 5$ oraz dla $n \geq 1$ $a_{n+2}=a_{n+1} \cdot a_n $ (mod 10). Czy w takim ciągu może wystąpić cyfra:
	
		\begin{itemize}
			\item $5$?
			\item $7$?
			\item $9$?
		\end{itemize}
	}
	
	\question {
		Zdefiniujmy ciąg cyfr w taki sposób, że na $n$-tym miejscu ciągu znajduje się pierwsza cyfra rozwinięcia dziesiętnego liczby $2^n$. Czyli początek ciągu to $1, 2, 4, 8, 1, 3, 6, \ldots$ Ciąg ten składa się z $9$ różnych cyfr. Będziemy rozważać częstotliwość występowania każdej z nich (czyli stosunek wystąpień danej cyfry do określonego miejsca w ciągu do łącznej liczby cyfr do tego miejsca)
	
		\begin{itemize}
			\item Dla dostatecznie długiego ciągu wszystkie cyfry będą występować równie często.
			\item Cyfra $2$ będzie występować częściej niż cyfra $3$.
			\item Cyfra $1$ będzie występować ponad dwukrotnie częściej niż cyfra $9$.
		\end{itemize}
	}
	
	\question {
		Czy liczba, której przedstawienie w systemie binarnym to $1011000011$ jest:
	
		\begin{itemize}
			\item podzielna przez $3$?
			\item dzielnikiem $10110000111$ (system binarny)?
			\item w systemie czwórkowym postaci $23003$?
		\end{itemize}
	}
	
	\question {
		Jeżeli funkcja $f: \R \rightarrow \R$, spełnia dla każdej liczby rzeczywistej $f(x) = f(f(x)) + x$, to:
	
		\begin{itemize}
			\item $f$ jest różnowartościowa
			\item istnieje $x > 0$, taki że $f(f(x)) = x$
			\item $f^{42}(1) = 1$, przy czym $f^{k}(x) = f(f^{k-1}(x))$, $f^1(x) = f(x)$
		\end{itemize}
	}
	
	\question {
		Ciąg Fibonacciego to taki ciąg, że $F_0 = 0$, $F_1 = 1$ oraz $F_{n+2} = F_{n+1} + F_n$ dla $n$ całkowitych nieujemnych. Niech $f(n)$ będzie najmniejszą taką liczbę naturalną, że $n$ dzieli $F_{f(n)}$ oraz $F_{f(n)+1}-1$, lub równe 0 jeśli nie ma takiej liczby. Wówczas:
	
		\begin{itemize}
			\item $f(5) = 10$
			\item $f(66) < f(88)$
			\item istnieje nieskończenie wiele takich $n$, że $f(n) = 0$
		\end{itemize}
	}
	
	\questionwithasterix{
		Jaś napotkał tablicę z liczbami naturalnymi od 1 do 20 i postanowił zagrać w grę. W każdym ruchu ściera dwie liczby z tablicy i rysuje trójkąt prostokątny o przyprostokątnych tych długości, po czym dopisuje do liczb na tablicy długość wysokości opuszczonej na przeciwprostokątną. Które z tych nierówności może spełniać otrzymana na końcu liczba?
		
		\begin{itemize}
			\item $1 \leq x$
			\item $x \leq \frac{\sqrt{2}}{2}$
			\item $\frac{\sqrt{2}}{2} \leq x$
		\end{itemize}
	}
	
	\question {
		Na tablicy napisanych jest $97$ liczb postaci $\frac{49}{k}$ dla $1 \leq k \leq 97$. W każdym ruchu pewne dwie liczby $a$ i $b$ zosają zmazane i zastąpione przez liczbę $2ab-a-b+1$. Jaka liczba może pozostać na tablicy po $96$ krokach?
		
		\begin{itemize}
			\item $\frac{1}{2}$
			\item $1$
			\item $2$
		\end{itemize}
	}
	
	\question {
		Pod domem Ani zatrzymują się autobusy linii 1 i 2, pierwszy kursuje co 10 minut zaczynając od 10:05 i jeździ do sklepu z rogalikami o nadzieniu truskawkowym, a drugi co 20 minut od 12:00 i jeździ do sklepu z rogalikami o nadzieniu śliwkowym. \\
		Ania codziennie wychodzi na przystanek o losowej porze między 15:00 i 16:00, a następnie wsiada w pierwszy autobus który przyjedzie. \\
		Jakie jest prawdopodobieństwo, że Ania będzie czekała nie dłużej niż 5 minut oraz pojedzie do sklepu z rogalikami o nadzieniu truskawkowym?
		
		\begin{itemize}
			\item $\frac{1}{4}$
			\item $\frac{1}{2}$
			\item $\frac{3}{4}$
		\end{itemize}
	}
	
	\question {
		O jaki kąt zgodnie ze wskazówkami zegara nalezy obrócić parabolę o równaniu $y=x^2$, żeby miała miejsce zerowe dla $x=2 \sqrt{3}$.
		
		\begin{itemize}
			\item $60^{\circ}$  
			\item $45^{\circ}$
			\item $30^{\circ}$
		\end{itemize}
	}
	
	\question {
		Ile liczb całkowitych z przedziału $[1,2022]$ można przedstawić jako różnicę kwadratów dwóch liczb całkowitych. 
		
		\begin{itemize}
			\item więcej niż $1011$
			\item więcej niż $1516$
			\item więcej niż $2000$
		\end{itemize}
	}
	
	\questionwithasterix {
		Liczba $\sqrt{13+2 \sqrt{12}}-\sqrt{3(7+\sqrt{48})}$ jest:
		
		\begin{itemize}
			\item ujemna.
			\item całkowita.
			\item niewymierna.
		\end{itemize}
	}
	
\end{enumerate}

\end{document}