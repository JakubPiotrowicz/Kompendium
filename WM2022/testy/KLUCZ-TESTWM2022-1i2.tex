\documentclass[12pt, a4paper]{article}
\usepackage[margin=2.5cm]{geometry}
\usepackage{fancyhdr}
\usepackage[MeX]{polski}
\usepackage[utf8]{inputenc} 
\usepackage[T1]{fontenc}
\usepackage{amsmath}
\usepackage{amsfonts}
\usepackage{amssymb}
\usepackage{amsthm}
\newcommand{\Z}{\mathbb{Z}}
\newcommand{\R}{\mathbb{R}}
\newcommand{\N}{\mathbb{N}}
\newcommand{\Q}{\mathbb{Q}}
\newcommand{\NWD}{\text{NWD}}
\newcommand{\NWW}{\text{NWW}}
\newcommand{\question}[1]{\normalitem \begin{samepage}#1 \end{samepage}}
\newcommand{\questionwithasterix}[1]{ \asterixitem \begin{samepage}#1 \vspace{6cm}\end{samepage}}
\usepackage{color}
\usepackage{pifont}
\newcommand{\cmark}{\textcolor{green}{T}}%
\newcommand{\xmark}{\textcolor{red}{N}}%
\newcommand{\yes}{\rlap{\framebox(15,15)} {\raisebox{2pt}{\large\hspace{-1pt}\cmark}}%
\hspace{3pt}}
\newcommand{\no}{\rlap{\framebox(15,15)} {\raisebox{2pt}{\large\hspace{-1pt}\xmark}}%
\hspace{3pt}}
\cfoot{\thepage}

\begin{document}

\begin{center}
\noindent \textbf{Prosimy wypełnić poniższe pola DRUKOWANYMI literami:}
\vspace{0.5cm}
\par Imię i nazwisko
\par $\begin{array}{|c|c|c|c|c|c|c|c|c|c|c|c|c|c|c|c|c|c|c|c|c|c|c|c|c|c|c|c|c|c|c|c|c|c|c|c|}
\hline
\makebox[0.1cm]{}&\makebox[0.1cm]{}&\makebox[0.1cm]{}&\makebox[0.1cm]{}&\makebox[0.1cm]{}&\makebox[0.1cm]{}&\makebox[0.1cm]{}&\makebox[0.1cm]{}&\makebox[0.1cm]{}&\makebox[0.1cm]{}&\makebox[0.1cm]{}&\makebox[0.1cm]{}&\makebox[0.1cm]{}&\makebox[0.1cm]{}&\makebox[0.1cm]{}&\makebox[0.1cm]{}&\makebox[0.1cm]{}&\makebox[0.1cm]{}&\makebox[0.1cm]{}&\makebox[0.1cm]{}&\makebox[0.1cm]{}&\makebox[0.1cm]{}&\makebox[0.1cm]{}&\makebox[0.1cm]{}&\makebox[0.1cm]{}&\makebox[0.1cm]{}&\makebox[0.1cm]{}&\makebox[0.1cm]{}&\makebox[0.1cm]{}&\makebox[0.1cm]{}&\makebox[0.1cm]{}&\makebox[0.1cm]{}&\makebox[0.1cm]{}&\makebox[0.1cm]{}&\makebox[0.1cm]{}&\makebox[0.1cm]{}\\
\hline
\end{array}$
\vspace{0.2cm}
\par E-mail
\par $\begin{array}{|c|c|c|c|c|c|c|c|c|c|c|c|c|c|c|c|c|c|c|c|c|c|c|c|c|c|c|c|c|c|c|c|c|c|c|c|}
\hline
\makebox[0.1cm]{}&\makebox[0.1cm]{}&\makebox[0.1cm]{}&\makebox[0.1cm]{}&\makebox[0.1cm]{}&\makebox[0.1cm]{}&\makebox[0.1cm]{}&\makebox[0.1cm]{}&\makebox[0.1cm]{}&\makebox[0.1cm]{}&\makebox[0.1cm]{}&\makebox[0.1cm]{}&\makebox[0.1cm]{}&\makebox[0.1cm]{}&\makebox[0.1cm]{}&\makebox[0.1cm]{}&\makebox[0.1cm]{}&\makebox[0.1cm]{}&\makebox[0.1cm]{}&\makebox[0.1cm]{}&\makebox[0.1cm]{}&\makebox[0.1cm]{}&\makebox[0.1cm]{}&\makebox[0.1cm]{}&\makebox[0.1cm]{}&\makebox[0.1cm]{}&\makebox[0.1cm]{}&\makebox[0.1cm]{}&\makebox[0.1cm]{}&\makebox[0.1cm]{}&\makebox[0.1cm]{}&\makebox[0.1cm]{}&\makebox[0.1cm]{}&\makebox[0.1cm]{}&\makebox[0.1cm]{}&\makebox[0.1cm]{}\\
\hline
\end{array}$
\vspace{0.2cm}
\par \hspace{1.7cm} Nr telefonu \hspace{2.5cm} Klasa
\par $\begin{array}{|c|c|c|c|c|c|c|c|c|c|c|c|}
\hline
\makebox[0.1cm]{+}&\makebox[0.1cm]{4}&\makebox[0.1cm]{8}&\makebox[0.1cm]{}&\makebox[0.1cm]{}&\makebox[0.1cm]{}&\makebox[0.1cm]{}&\makebox[0.1cm]{}&\makebox[0.1cm]{}&\makebox[0.1cm]{}&\makebox[0.1cm]{}&\makebox[0.1cm]{}\\
\hline
\end{array}$
\hspace{1cm}
$\begin{array}{|c|c|}
\hline
\makebox[0.1cm]{}&\makebox[0.1cm]{}\\
\hline
\end{array}$
\end{center}\vspace{0.5cm}
	\begin{center}{\hspace{1cm}{\textbf{Klucz do testu kwalifikacyjnego na Warsztaty Matematyczne 2022}}}
				\newline \newline Klasy pierwsze i drugie\end{center}
\par
\vspace{0.3cm}
Test składa się z uporządkowanych w kolejności \underline{\textbf{losowej}} 30 zestawów po 3 pytania. Na pytania odpowiada się ,,tak'' lub ,,nie'' poprzez wpisanie odpowiednio ,,\textbf{T}'' bądź ,,\textbf{N}''
w pole obok pytania. W danym trzypytaniowym zestawie możliwa jest dowolna kombinacja
odpowiedzi ,,tak'' i ,,nie''. W zestawach zaznaczonych gwiazdką (gwiazdka wygląda tak: *)
prócz udzielenia odpowiedzi należy je uzasadnić. Test trwa 180 minut.
\vspace{0.5cm}
\par
\textbf{Zasady punktacji}
\begin{itemize}
   \item Za pojedynczą poprawną odpowiedź: \textbf{1} punkt.
   \item Za pojedynczą niepoprawną odpowiedź: \textbf{-1} punkt.
   \item Za brak odpowiedzi: \textbf{0} punktów.
   \item Za zadanie zrobione w całości dobrze dodatkowe \textbf{2} punkty.
   \item Za poprawne uzasadnienie pojedynczej odpowiedzi: \textbf{1} punkt.
   \item Za niepoprawne uzasadnienie pojedynczej odpowiedzi bądź brak takowego: \textbf{0} pkt.
\end{itemize}
	\begin{center}Powodzenia!\end{center}
Uwaga! Przez zbiór liczb naturalnych w zadaniach rozumiemy zbiór liczb całkowitych większych lub równych $0$.
\vspace{0.2cm}

\renewcommand{\labelitemi}{\LARGE{$\square$}}
\newcommand{\normalitem}{\stepcounter{enumi}\item[\textbf{\theenumi. }]}
\newcommand{\asterixitem}{\stepcounter{enumi}\item[\textbf{\theenumi*. }]}

\begin{enumerate}
    \question {
        Dana jest kwadratowa kartka $ABCD$, niech $E, F, G, H$ będą środkami odcinków $AB, BC, CD, DA$ odpowiednio.
        Zginamy kartkę wzdłuż prostej $FH$, tak aby punkt $A$ przeszedł na punkt $D$, następnie wzdłuż prostej $EG$ tak aby punkt $D$ przeszedł na punkt $C$, następnie wzdłuż prostej $FG$, tak aby środek $ABCD$ przeszedł na punkt $C$. Tak złożoną kartkę tniemy równolegle do $CD$ przechodząc przez środek $CF$.

		\begin{itemize}
			\item [\no]Czy dostaniemy $4$ kawałki papieru?
			\item [\yes]Czy dokładnie $2$ kawałki papieru będą prostokątami?
			\item [\no]Czy dokładnie 4 kawałki papieru będą trójkątami?
		\end{itemize}
	}
	
	\questionwithasterix {
		Dany jest trójkąt $ABC$ o bokach: $|AB| = 13, |BC| = 12, |AC| = 5$. 
        $D$ - spodek wysokości z $C$, niech $E$ będzie punktem przecięcia kwadratu $ABFG$ (zawierającego w sobie punkt $C$) z półprostą $DC$. \\
        Jaki jest stosunek $\frac{[ACEG]}{[CBFE]}$?

		\begin{itemize}
			\item [\yes]$\frac{25}{144}$
			\item [\no]$\frac{5}{12}$
			\item [\no]$\frac{12}{5}$
		\end{itemize}
	}
	
	\questionwithasterix{
		Martyna i Oliwia grają w grę.
        Zaczyna Martyna, na początku mając na tablicy liczbę $2$. Dziewczyny wykonują ruchy na przemian, w każdym ruchu muszą dodać jakiś dzielnik właściwy aktualnej liczby do zapisanej liczby. Wygrywa ta z dziewczyn, która napisze liczbę większą lub równą $x$. \\
        Dzielniki właściwe liczby $10$ to: $1, 2, 5$. \\
        Czy Martyna ma strategię wygrywającą dla $x$ równego:

		\begin{itemize}
			\item [\no]$4$?
			\item [\yes]$9$?
			\item [\yes]$1237$?
		\end{itemize}
	}
	
	\question{
		Czy
		
		\begin{itemize}
			\item [\no]nwd$(8649,8789) > 100$
			\item [\yes]nwd$(8917,7471)>100$
			\item [\yes]nwd$(7164,8358)>100$
		\end{itemize}
	}
	
	\question{
	    Ile cyfr ma najmniejsza liczba, która kończy się na $6$ i jeśli się jej ostatnią cyfrę ($6$) przeniesie na początek, to otrzymana liczba jest $4$ razy większa od liczby początkowej?

		\begin{itemize}
			\item [\yes]$6$
			\item [\no]$8$
			\item [\no]$10$
		\end{itemize}
	}
	
	\question {
		Liczba $139$:

		\begin{itemize}
			\item[\yes] Jest pierwsza.
			\item[\no] Może być przedstawiona jako suma dwóch kwadratów liczb całkowitych.
			\item [\no]Może być przedstawiona jako suma dwóch sześcianów liczb całkowitych.
		\end{itemize}
	}
	
	\question {
		Dany jest trójkąt równoboczny o boku $1$ i prostokątny o przyprostokątnej $1$.

    	\begin{itemize}
			\item [\no]trójkąt równoboczny ma większy promień okręgu wpisanego.
			\item [\no]trójkąt równoboczny ma większy promień okręgu opisanego
			\item [\no]istnieje dokładnie jeden wielościan, którego wszystkie ściany to trójkąty równoboczne.
		\end{itemize}
	}
	
	\question {
		Na boku BC trójkąta ABC, spełniającego kąt $ACB = 170^{\circ}$, obrano taki punkt $D$, że $BD = AC$. Niech $P$ i $M$ będą odpowiednio środkami odcinków $CD$ i $AB$. Miara kąta $BPM$ wynosi:

		\begin{itemize}
			\item [\yes]$85^{\circ}$
			\item [\no]$90^{\circ}$
			\item [\no]$95^{\circ}$
		\end{itemize}
	}
	
	\question {
		Mamy dany ciąg liczb naturalnych od $1$ do $16$.

		\begin{itemize}
			\item [\yes]Możemy podzielić te liczby w pary, tak aby suma każdej z nich była kwadratem liczby całkowitej.
			\item [\yes]Możemy ustawić je w szeregu, tak aby suma każdych dwóch kolejnych była kwadratem liczby całkowitej.
			\item [\no]Możemy ustawić je w kole, tak aby suma każdych dwóch kolejnych była kwadratem liczby całkowitej.
		\end{itemize}
	}
	
	\questionwithasterix {
		Oceń prawdziwość podanych relacji:

		\begin{itemize}
			\item [\no]$\frac{1}{3} + \frac{1}{15} + \frac{1}{35} + \frac{1}{63} + \frac{1}{99} + \frac{1}{143} = \frac{7}{13}$.
			\item [\no]Dla dowolnego $n$: $\frac{1}{1} + \frac{1}{2} + \frac{1}{3} + \frac{1}{4} + \ldots + \frac{1}{n} < 1000$.
			\item [\yes] Dla dowolnego $n$: $\sum_{i=1}^n \frac{i}{1+i^2+i^4} < \frac{1}{2}$ 
		\end{itemize}
	}
	
	\question{
		Dane są takie liczby całkowite dodatnie $a, b, x,$ że:
        $$x \hspace{0.1cm} | \hspace{0.1cm} a+5b$$
        $$x \hspace{0.1cm} | \hspace{0.1cm} 3a+b $$
        Czy z tych warunków wynika, że:
        
		\begin{itemize}
			\item [\yes]$x \hspace{0.1cm} | \hspace{0.1cm} 29a^2+38ab+53b^2$
			\item [\yes]$x \hspace{0.1cm} | \hspace{0.1cm} 7a^2+21b^2$
			\item [\no]$x \hspace{0.1cm} | \hspace{0.1cm} -8a^2+4ab+22b^2$
		\end{itemize}
	}
	
	\questionwithasterix{
		Każdy punkt płaszczyzny pomalowano na pewien z $k$ różnych kolorów
        (każdy kolor został użyty). Prawdą jest, że;
        
		\begin{itemize}
			\item [\yes]jeśli $k = 3$ to zawsze istnieją dwa punkty tego samego koloru odległe o $1$.
			\item [\yes]jeśli $k = 4$ to zawsze istnieją dwa punkty tego samego koloru odległe o $1$ lub $\sqrt{3}$.
			\item [\no]jeśli $k = 4$ to istnieje takie kolorowanie, że każda prosta jest
                jednokolorową lub dwukolorowa.
		\end{itemize}
	}
	
	\question{
	    Dany jest turniej - każdy zawodnik rozgrywa dokładnie jeden mecz z każdym innym
        i nie ma remisów. Mistrzem turnieju nazwiemy zawodnika, który dla każdego zawodnika $A$, wygrał z nim lub kimś kto wygrał z $A$. Czy:

		\begin{itemize}
			\item [\no]w turnieju może być dokładnie 2 mistrzów.
			\item [\yes]w turnieju może być dokładnie 3 mistrzów.
			\item [\no]w czteroosobowym turnieju może zdarzyć się, że każdy jest mistrzem.
		\end{itemize}
	}
	
	\question {
		Niech $d_1, d_2, \ldots, d_m$ oznaczają wszystkie dodatnie dzielniki $n$ oraz niech \\
        $\sigma(n) = d_1 + d_2 + \ldots + d_m$. Czy:

		\begin{itemize}
			\item [\yes]$\frac{\sigma(n)}{(\frac{1}{d_1} + \frac{1}{d_2} + \ldots + \frac{1}{d_m})} = n$
			\item [\no]$\sigma(2^k \cdot n) = (2^{k+1}-1) \cdot \sigma(n)$
			\item [\no]istnieją dokładnie $2$ takie liczby parzyste $n$, że $\sigma(n) = 2n$ oraz $n < 1000$
		\end{itemize}
	}
	
	\questionwithasterix {
        Na pewnej wyspie żyją dwa typy mieszkańców: prawdomówni - którzy zawsze mówią prawdę i kłamcy, którzy zawsze kłamią. Po przybyciu na wyspę podróżnik spotkał dwóch mieszkańców: wysokiego i niskiego. Zapytał wysokiego, czy obaj są prawdomówni, ale z jego wypowiedzi nie można było wywnioskować, kim oni byli. Wówczas zapytał niskiego, czy wysoki jest prawdomówny, a gdy ten odpowiedział, podróżnik wiedział, do jakiego typu należał każdy z nich. Czy napotkani mieszkańcy mogli być:
        
        \begin{itemize}
			\item [\no]obaj prawdomówni
			\item [\no]obaj kłamcami
			\item [\yes]niski prawdomówny, zaś wysoki kłamcą 
		\end{itemize}
	}
	
	\question {
		W tym zadaniu $d$ oznacza długość średnicy podstawy stożka, zaś $l$ - długość jego tworzącej. Czy można zbudować stożek, gdy:
		
		\begin{itemize}
			\item [\yes]$d = 6$, $l = 5$ 
			\item [\no]$d = 12$, $l = 5$
			\item [\yes]$d = 22$, $l = 12$
		\end{itemize}
	}
	
	\questionwithasterix{
		Niech $A=(0,0)$, $B=(1,0)$, $C=(2,0)$, $D=(3,0)$, $E=(0,1)$. Czy
	
		\begin{itemize}
			\item [\yes]$\measuredangle BED = \measuredangle ECA$?
			\item [\yes]$\measuredangle BEC = \measuredangle EDA$?
			\item [\yes]$\measuredangle AEB = \measuredangle EBA$?
		\end{itemize}
	}
	
	\question {
		Czy liczba $3^{105} + 4^{105}$ jest podzielna przez:
		
		\begin{itemize}
			\item [\no]5?
			\item [\no]11?
			\item [\yes]13?
		\end{itemize}
	}
	
	\question {
		Pewien pijak spacerując po nadmorskich klifach znalazł się trzy kroki od przepaści (trzeci już wpada w przepaść). Jest on pijany, dlatego wykonuje losowe ruchy przybliżając się o krok do przepaści z prawdopodobieństwem
        $\frac{2}{5}$ oraz oddalając się o krok od przepaści z prawdopodobieństwem $\frac{3}{5}$. Zakładamy, że pijak może odejść od przepaści dowolnie daleko oraz nie kończy spaceru za wyjątkiem upadku w przepaść.

		\begin{itemize}
			\item [\no]Prawdopodobieństo pozostania żywym po pięciu krokach wynosi więcej niż $0.9$.
			\item [\no]Pijak może spaść w $24$ kroku.
			\item [\no]Po nieskończenie długim czasie szansa na przeżycie pijaka wynosi $\frac{4}{5}$.
		\end{itemize}
	}
	
	\question {
		Codziennie zaraz po wyjściu ze szkoły Jaś idzie na stację metra i wsiada w pierwszy pociąg, który nadjedzie, niezależnie od kierunki jazdy. Na Kabatach mieszka babcia Jasia, a
        niedaleko Młocin jego dziewczyna. Jaś zawsze korzysta z okazji i odwiedza osobę, w pobliżu której się znalazł. Zakładamy, że pociągi metra kursują w stałych odstępach czasu, z równą częstotliwością
        w każdym z dwóch kierunków.

		\begin{itemize}
			\item [\no]Jaś zawsze z prawdopodobieństwem $\frac{1}{2}$ pojedzie do babci.
			\item [\yes]Czy może się zdarzyć, że Jaś bez celowego działania będzie jeździł do dziewczyny $5$ razy częściej, niż do babci?
			\item [\yes]Czy możliwe jest, że gdyby Jaś zmienił strategię i przepuszczał zawsze pierwszy napotkany pociąg metra i wsiadał do drugiego to zmieniłby częstotliwość widzenia babci? 
		\end{itemize}
	}
	
	\question {
		Dane są dwie liczby niewymierne.\\ 
		Czy możliwe jest, aby:

		\begin{itemize}
			\item [\yes]Ich suma była liczbą wymierną
			\item [\yes]Zarówno ich iloczyn, jak i iloraz były wymierne
			\item [\yes]Występowało między nimi nieskończenie wiele liczb wymiernych
		\end{itemize}
	}
	
	\question {
		Które z poniższych planszy można pokryć klockami $6 \times 1$?
		
		\begin{itemize}
			\item [\no]$16 \times 15$
            \item [\no]$19 \times 19$ bez środkowego pola
            \item [\yes]$13 \times 13$ bez środkowego pola
		\end{itemize}
	}
	
	\question {
		Zdanie ''Dla dowolnych $n$ kolejnych liczb naturalnych, można wybrać dwie z nich (niekoniecznie różne), tak że ich iloczyn daje resztę $1$ w dzieleniu przez $m$'' jest prawdziwe dla:
		
		\begin{itemize}
			\item [\yes]$m = 8$, $n = 2$
            \item [\no]$m = 13$, $n = 5$
            \item [\yes]$m = 17$, $n = 8$
		\end{itemize}
	}
	
	\question {
	    Dane są: $f(x) = x^2 + ax + b$, $g(x) = x^2 + cx + d$, takie, że: $a \neq c$, $b \neq d$, \\
	    Spełniony jest warunek: $f(17)+f(105) = g(17) + g(105)$. \\
	    Ile rozwiązań ma równanie $f(x) = g(x)$?
	    
	    \begin{itemize}
	        \item [\no]0
	        \item [\yes]1
	        \item [\no]2
	    \end{itemize}
	}
	
	\question {
	    W kwadracie $ABCD$ wybrano 2 losowe punkty na boku $CD$: $M, N$.
	    Trójkąty $ABM, ABN$ na pewno:
	    
	    \begin{itemize}
	        \item [\no]są podobne.
	        \item [\no]mają równe obwody.
	        \item [\yes]mają równe pola.
	    \end{itemize}
    }
    
	\question {
	    Mając pierścień z dwóch okręgów o wspólnym środku, cięciwa większego okręgu styczna do mniejszego ma długość k. Pole między okręgami wyraża się przez:
	    
	    \begin{itemize}
	        \item [\no]$\pi k^2$
	        \item [\yes]$\frac{\pi k^2}{4}$
	        \item [\no]$\frac{\pi k}{4}$
	    \end{itemize}
	}
	
	\questionwithasterix {
		Liczba $\sqrt{12+2 \sqrt{27}}-\sqrt{19-4\sqrt{12}}$ jest:
		
		\begin{itemize}
			\item [\no]ujemna.
			\item [\no]całkowita.
			\item [\yes]niewymierna.
		\end{itemize}
	}
	
	\question{
	    Liczba $a^3b^5c^2$ jest siódmą potęgą pewnej liczby całkowitej, gdzie $a,b$ i $c$ są liczbami całkowitymi. Czy z tego wynika, że siódmą potęgą jest również liczba:
	    
	    \begin{itemize}
	        \item [\yes]$ab^4c^3$?
	        \item [\no]$a^2bc$?
	        \item [\yes]$a^{22}b^{333}c^{55555}$?
	    \end{itemize}
	}
	
	\question{
	    Mamy daną liczbę naturalną $k$. Ile uporządkowanych rozwiązań $(m,n)$ może mieć równanie $2^m+2^n=k$, gdzie $m$ i $n$ są liczbami całkowitymi.
	    
	    \begin{itemize}
	        \item [\yes]0
	        \item [\yes]1
	        \item [\yes]2
	    \end{itemize}
	}
	
	\question {
		Czy istnieje $100$ kolejnych liczb naturalnych wśród których:
		
		\begin{itemize}
			\item [\no]Dokładnie $12$ jest liczbami Fibbonacciego?
			\item [\yes]Dokładnie $7$ jest liczbami pierwszymi?
			\item [\yes]Dokładnie $7$ jest potęgami dwójki o całkowitym wykładniku?
		\end{itemize}
		
	}
	
\end{enumerate}

\end{document}